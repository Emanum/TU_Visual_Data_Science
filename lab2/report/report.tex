\documentclass[11pt]{article}

% If you're new to LaTeX, here's some short tutorials:
% https://www.overleaf.com/learn/latex/Learn_LaTeX_in_30_minutes
% https://en.wikibooks.org/wiki/LaTeX/Basics

% Formatting
\usepackage[utf8]{inputenc}
\usepackage[margin=1in]{geometry}
\usepackage[titletoc,title]{appendix}
\usepackage[parfill]{parskip}

\usepackage{subcaption}
\captionsetup[subfigure]{list=true, font=large, labelfont=bf, 
labelformat=brace, position=top}

% Math
% https://www.overleaf.com/learn/latex/Mathematical_expressions
% https://en.wikibooks.org/wiki/LaTeX/Mathematics
\usepackage{amsmath,amsfonts,amssymb,mathtools}

% Images
% https://www.overleaf.com/learn/latex/Inserting_Images
% https://en.wikibooks.org/wiki/LaTeX/Floats,_Figures_and_Captions
\usepackage{graphicx,float}

% Tables
% https://www.overleaf.com/learn/latex/Tables
% https://en.wikibooks.org/wiki/LaTeX/Tables

% Algorithms
% https://www.overleaf.com/learn/latex/algorithms
% https://en.wikibooks.org/wiki/LaTeX/Algorithms
\usepackage[ruled,vlined]{algorithm2e}
\usepackage{algorithmic}
\usepackage{listings}

% Code syntax highlighting
% https://www.overleaf.com/learn/latex/Code_Highlighting_with_minted
%\usepackage{minted}
%\usemintedstyle{borland}

% References
% https://www.overleaf.com/learn/latex/Bibliography_management_in_LaTeX
% https://en.wikibooks.org/wiki/LaTeX/Bibliography_Management
\usepackage{biblatex}
%\addbibresource{references.bib}

% Title content
\title{Visual Data Science - Lap Part 2}
\author{Manuel Eiweck - 01633012}
\date{January 15, 2021}

\begin{document}

\maketitle

% Abstract
%\begin{abstract}
%    Add your abstract here.
%\end{abstract}

% Introduction and Overview
\section{Dataset}

I did choose a dataset containing PC games from the popular store Steam. The dataset was created around May 2019 and contains ~27.000 game entries with each one having 18 attributes.
However, the dataset is not guaranteed to be fully complete. In addition, it has been already cleaned from unreleased titles and non-games like software. The original data source was the official Steam API and SteamSpy a website which collects all kind of stats from steam.

%\subsection{Data preparation}

%As the field 'platforms','categories' and 'genres' can contain multiple values I split these columns into one column for each different value. Example: platforms has the values windows, linux and mac, after the preparation I now have 3 columns as a bit mask representing if the game supports that platform. Furthermore, 'owners' has a lower and upper bound therefore I also split that into separated columns. As our data contains a column price we add a cloumn type which is either 'Free' or 'Paid' depending on the value of price is 0 or not.

%TODO: score rating
%\bigbreak
%Steam Rating
%https://steamdb.info/blog/steamdb-rating/

\section{Insights}

\subsection{Insight - generated playtime Free to Play vs Paid games}

\subsubsection{What is the interesting fact you found in the data?}

I wanted to compare how the total playtime of all games is split between free-to-play and paid games.\\
I found out that about 60\% percentage comes from free to play games while 40\% comes from paid games. Furthermore, I also found out that there are 3 big games who generate way more playtime alone vs thousands of games together. 

\subsubsection{Give a brief description how you discovered the
insight.}

First I added a column 'Total Playtime'. Where Total Playtime is defined by:
\smallbreak
$Total Playtime = average Playtime * owners Lower Bound.$
\smallbreak
Then I created different plots and percentage evaluations to discover the distribution and confirm the found insights. 

Figure \ref{fig:insight1_1} shows a bar chart of all games. We can see that Free games generate a more playtime than paid games.
Figure \ref{fig:insight1_2} shows a bar chart of games separated per release year. This allows us first group our games to see if we can detect some distributions. Interesting is that games released in 2012, 2013 and 2017 seems to be very popular. 
\begin{figure*}
    \centering
    \begin{subfigure}[b]{0.475\textwidth}
        \centering
        \includegraphics[width=1\textwidth]{graphics/insight1_graph1.png}
        \caption{Distribution by type.}
        \label{fig:insight1_1}
    \end{subfigure}
    \hfill
    \begin{subfigure}[b]{0.475\textwidth}
        \centering
        \includegraphics[width=1\textwidth]{graphics/insight1_graph2.png}
        \caption{Distribution by release year and type.}
        \label{fig:insight1_2}
    \end{subfigure}
    \vskip\baselineskip
    \begin{subfigure}[b]{0.475\textwidth}
        \centering
        \includegraphics[width=1\textwidth]{graphics/insight1_graph3.png}
        \caption{Distribution of the top 10 most played games.}
        \label{fig:insight1_3}
    \end{subfigure}
    \label{fig:insight1}
    \caption{Distribution of the total playtime generated by all games.}
\end{figure*}

Figure \ref{fig:insight1_3} shows a bar of the top 10 games ranked by total playtime. We can see that Dota2 released  2013, PlayerUnknown's Battlegrounds released 2017 and Counter-Strike: Global Offensive released 2012 have combined way more total playtime than the rest.\\
This explains the spikes 2012, 2013 and 2017 we saw in Figure \ref{fig:insight1_2}.

\subsubsection{How did you test the statistical significance of the insight?}

(i) Hypothesis:\\
1. Free to play games have more total Playtime than Paid games.\\
2. The top 3 Games are played way more than all other games together.

(ii)Tests:\\
As I have all the data available I decided to just use the calculated 
percentages

(iii)Check hypothesis:\\
Free to Play: 61\% vs. Paid: 39\% So we can assume that the first hypothesis is correct.\\
Playtime of top 3 games:  60\% rest: 40\% So we can assume that the second hypothesis is correct.

\subsection{Insight - Rating distribution}

\subsubsection{What is the interesting fact you found in the data?}

I found out that games with a more recent release date tend to have a lower rating than games with an older release date.\\

\subsubsection{Give a brief description how you discovered the
insight.}

As the dataset only includes the count of positive and negative reviews I used the scoring Method from "SteamDB.info" https://steamdb.info/blog/steamdb-rating/ to calculate a score from 0-1. The basic concept is to pull the score stronger towards 0.5 the less total reviews there are. So it should compensate for small  games and make them comparable to large games. 

I first created a scatter plot (see Figure \ref{fig:insight2_1}) with all games and their rating score and release date. To make sure this was not only some noise in the data I created additional plots. I decided to compare the number of released games per year from the complete dataset with released games from the top 10 percent rated games. Figure \ref{fig:insight2_2} shows the distribution of the games along the years. Additional, figure \ref{fig:insight2_3} shows the percentage of the released games being in the top 10 percent rated games of all years.  

\begin{figure}
    \centering
    \begin{subfigure}[b]{1\textwidth}
        \centering
        \includegraphics[width=1\textwidth]{graphics/insight2_graph1.png}
        \caption{Scatter plot for all games with rating and release date}
        \label{fig:insight2_1}
    \end{subfigure}
    \vskip\baselineskip
    \begin{subfigure}[b]{0.475\textwidth}
        \centering
        \includegraphics[width=1\textwidth]{graphics/insight2_graph2.png}
        \caption{Bar chart of all released games and the distribution of the top 10\% rated games.}
        \label{fig:insight2_2}
    \end{subfigure}
    \hfill
    \begin{subfigure}[b]{0.475\textwidth}
        \centering
        \includegraphics[width=1\textwidth]{graphics/insight2_graph3.png}
        \caption{Trend of released games being part of the top 10\% rated games}
        \label{fig:insight2_3}
    \end{subfigure}
    \label{fig:insight2}
    \caption{Exploration of the rating distribution.}
\end{figure}

\subsubsection{How did you test the statistical significance of the insight?}

(i) Hypothesis:\\
Games with newer release date have a lower rating than older games.

(ii)Tests:\\
I decided to use the linear regression test with a trend line. I did this by adding the feature in the scatter plot in the dashboard. 

(iii)Check hypothesis:\\
Both lines show a falling trend so I assume the hypothesis is correct.

\subsection{Insight - statistics of platforms and genres}

\subsubsection{What is the interesting fact you found in the data?}

I found out that there are some categories which are flooded with game releases and mostly not played for a long time. There seems to be some kind of games people tend to play longer and therefore should be more attractive for game developer / publisher. 

\subsubsection{Give a brief description how you discovered the
insight.}

My basic concept was to write a dynamic script which allows me to quickly filter and compare different subsets of the data. This way I can simply try out various combinations until I find something interesting.\\
My first step is a script with dynamic filters. In the data there are columns (platform, categories, developer and publisher) with multiple values for each row. So I had to find a way to handle these. An example filter call could look like this:\\
\begin{minipage}{\linewidth}
\begin{lstlisting}
preFilteredDF = multipleFilter(df,[
            lambda row: orFilter(row,'platforms',['linux','mac']),
            lambda row: biggerFilter(row,'owners_low_bound',0),
            lambda row: smallerFilter(row,'price',5),
            lambda row: orFilter(row,'categories',['Single-player']),
            lambda row: notFilter(row,'categories',['Multi-player']),
        ])
\end{lstlisting}
\end{minipage}

Then I added the ability to split this filtered dataset into multiple subsets by a user defined column and list of combination the user wants to compare. For each subset either the mean, sum or variance is calculated and combined into a single dataframe. This dataframe can then be visualized in different graphs.

\begin{minipage}{\linewidth}
    \begin{lstlisting}
stats = getStatsDataFrame(
    preFilteredDF,  # our dataset from the first step
    'categories',   # the column we want to split the dataset
    [['Single-player'],['Multi-player']],
                    # the combinations we want to compare
    'or',           # how these filter combinations are applied 
                      (logical and, or, equals )
    'mean')         # kind of stats to be calculated

ax = stats['total_playtime'].plot(kind='bar', stacked=False)
ax.set_ylabel('total_playtime')
ax.set_title("Distribution")
    \end{lstlisting}
\end{minipage}

First I looked into the distribution of games along different platforms (see Figure \ref{fig:insight3_distro_plat}) Not surprisingly windows is clearly the dominant platform. What is interesting is that there are nearly zero games which don't support windows however the number of games supporting all platforms in comparison is quit large. 
I suspect that the main target platform for games is windows. In some cases game engines that makes it easy for developers to port there games to mac or linux are used. This would also explain why there are more releases for both platforms than for windows+linux or windows+mac.

\begin{figure}[h]
    \centering
    \begin{subfigure}[b]{0.45\columnwidth}
        \centering
        \includegraphics[width=\textwidth]{graphics/insight3_graph1.png}
        \label{fig:insight3_1}
    \end{subfigure}
    \begin{subfigure}[b]{0.45\columnwidth}
        \centering
        \includegraphics[width=\textwidth]{graphics/insight3_graph2.png}
        \label{fig:insight3_2}
    \end{subfigure}
    
    \caption{Distribution of games}
    \label{fig:insight3_distro_plat} 
  \end{figure}

  Additionally, I explored some statistics for the genre. In Figure \ref{fig:insight3_3} we can see that Single-player, steam trading cards and steam achievements seems popular for games. But when we look at the mean average playtime in Figure \ref{fig:insight3_4} we can see that these genres generate only small amounts of average playtime. MMO games on the other hand are not often released, but these games seems to be played way longer. 
  Valve anti cheat and includes source SDK has probably a high playtime as some popular games are using these and maybe pushing up the stats. 

  \begin{figure}[h]
    \centering
    \begin{subfigure}[h]{0.49\textwidth}
        \centering
        \includegraphics[width=\textwidth]{graphics/insight3_graph3.png}
        \caption{total number of games}
        \label{fig:insight3_3}
    \end{subfigure}
    \hfill
    \begin{subfigure}[h]{0.49\textwidth}
        \centering
        \includegraphics[width=\textwidth]{graphics/insight3_graph4.png}
        \caption{mean average playtime, x-label shows the number of datasets in ()}
        \label{fig:insight3_4}
    \end{subfigure}
    \vskip\baselineskip
    \begin{subfigure}[h]{0.49\textwidth}
        \centering
        \includegraphics[width=\textwidth]{graphics/insight3_graph6.png}
        \caption{Mean of the owners per genres}
        \label{fig:insight3_6}
    \end{subfigure}
    \caption{Statistics of different genres. The data only contains games with min. 10000 owners and excludes explicitly the top 3 games (Dota2, PUBG, CS:GO)}
    \label{fig:insight3_gen} 
  \end{figure}

\section{Comparison exploration vs. confirmation}

As I had never done any data exploration before and had to get familiar with the tools, my processes was the following: I began with preparing the data and calculating additional columns based on my knowledge and assumptions of the dataset. Furthermore, I generated a lot of bar chart distributions with different Filter to get a better understanding of the data and what columns and games are different from other and maybe ruin my plots as we saw the top 3 games are not really comparable to the rest.
Then I decided to analyze different combinations with the scatter plot in the dashboard where I found a possible correlation between release date and rating. Lastly I build the scripts to quickly filter and compare statistics of different subsets. My process of statistical confirmation was mainly to compare absolute numbers of sums, means, percentage of games fulfilling certain conditions.\\
However not all insights turned out to be true when considering multiple graphs and values. 

\section{Dashboard}

The target audience for my dashboard are people with an interest for gaming and current/historical trends in the industry. 
So I designed the dashboard with a bit of background knowledge about steam and online game stores in general in mind. 
The goal is not to present some particular already known facts for a specific reason but more to let people discover on their one. I want to give the users the tools they need to answer their own questions.

Therefore, the dashboard supports the dynamically filtering I also used during my analysis phase. The users can define their own filter queries, sort and limit the displayed items. The resulting dataset is then presented in different graphs, each one able to change the displayed axis, columns, size and color encoding. As a starting point I also included some of my own plots used for the report in a separated tab. This should give the users a first impression of the data. 


\end{document}
