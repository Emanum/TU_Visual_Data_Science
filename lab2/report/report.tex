\documentclass[11pt]{article}

% If you're new to LaTeX, here's some short tutorials:
% https://www.overleaf.com/learn/latex/Learn_LaTeX_in_30_minutes
% https://en.wikibooks.org/wiki/LaTeX/Basics

% Formatting
\usepackage[utf8]{inputenc}
\usepackage[margin=1in]{geometry}
\usepackage[titletoc,title]{appendix}
\usepackage[parfill]{parskip}

% Math
% https://www.overleaf.com/learn/latex/Mathematical_expressions
% https://en.wikibooks.org/wiki/LaTeX/Mathematics
\usepackage{amsmath,amsfonts,amssymb,mathtools}

% Images
% https://www.overleaf.com/learn/latex/Inserting_Images
% https://en.wikibooks.org/wiki/LaTeX/Floats,_Figures_and_Captions
\usepackage{graphicx,float}

% Tables
% https://www.overleaf.com/learn/latex/Tables
% https://en.wikibooks.org/wiki/LaTeX/Tables

% Algorithms
% https://www.overleaf.com/learn/latex/algorithms
% https://en.wikibooks.org/wiki/LaTeX/Algorithms
\usepackage[ruled,vlined]{algorithm2e}
\usepackage{algorithmic}

% Code syntax highlighting
% https://www.overleaf.com/learn/latex/Code_Highlighting_with_minted
%\usepackage{minted}
%\usemintedstyle{borland}

% References
% https://www.overleaf.com/learn/latex/Bibliography_management_in_LaTeX
% https://en.wikibooks.org/wiki/LaTeX/Bibliography_Management
\usepackage{biblatex}
%\addbibresource{references.bib}

% Title content
\title{Visual Data Science - Lap Part 2}
\author{Manuel Eiweck - 01633012}
\date{January 15, 2021}

\begin{document}

\maketitle

% Abstract
%\begin{abstract}
%    Add your abstract here.
%\end{abstract}

% Introduction and Overview
\section{Dataset}

I did choose a dataset containing PC games from the popular store Steam. The dataset was created around May 2019 and contains ~27.000 game entries with each one having 18 attributes.
However, the dataset is not guaranteed to be complete. In addition, it has been already cleaned from unreleased titles and non-games like software. The original data source was the official Steam API and SteamSpy a website which collects all kind of stats from steam.

\subsection{Data preparation}

As the field 'platforms','categories' and 'genres' can contain multiple values I split these columns into one column for each different value. Example: platforms has the values windows, linux and mac, after the preparation I now have 3 columns as a bit mask representing if the game supports that platform. Furthermore, 'owners' has a lower and upper bound therefore I also split that into separated columns. As our data contains a column price we add a cloumn type which is either 'Free' or 'Paid' depending on the value of price is 0 or not.

TODO: score rating
%\bigbreak
Steam Rating
https://steamdb.info/blog/steamdb-rating/

\section{Insight - Increasing number of Free to Play games}





\end{document}
